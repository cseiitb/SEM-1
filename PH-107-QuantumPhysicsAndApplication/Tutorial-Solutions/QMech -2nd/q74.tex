\documentclass{article}
\begin{document}
\title{QM Tutorial Sheet 3 Q.74}
\author{Ashwin Paranjape}
\maketitle



{\underline {\bf Solution}} : \\
\begin{center}
\end{center}

For $x<0$, $V=0 < E = 9V_0$. Hence wavefunction is given by
\[\phi_{1}(x)=Ae^{ik_1x}+Be^{-ik_1x}, k_1^2= \frac{2m(9V_0)}{\hbar^2}\]
For $0<x<d$, $V=5V_0 < E = 9V_0$. Hence wavefunction is given by
\[\phi_{2}(x)=Ce^{ik_2x}+De^{-ik_2x}, k_2^2= \frac{2m(4V_0)}{\hbar^2}\]
Irrespective of the third region potential, the boundry conditions here hold.
Thus at $x=0$, 
\[\phi_{1}(0) = \phi_{2}(0) \Rightarrow A+B=C+D\]
\[\phi'_{1}(0) = \phi'_{2}(0) \Rightarrow Ak_1-Bk_1=Ck_2 - Dk_2\]
%\[\Rightarrow C= (k_1 + k_2 )A/2 + (k_2 - k_1 ) B/2 \]
%\[\Rightarrow D= (k_2 - k_1 )A/2 + (k_2 + k_1 ) B/2\]


\[\phi_{3}(x)=Ee^{ik_3x}, k_3^2= \frac{2m(9-n)(V_0)}{\hbar^2}\] (No component in -ve x direction)
This holds for both $n>9, n<9 n = 9$
Solving at $x=d, k_2d=\pi$, we have

\[\phi_{2}(d) = \phi_{3}(d) \Rightarrow -C-D=Ee^{ik_3d}\]
\[\phi'_{2}(d) = \phi'_{3}(d) \Rightarrow - C + D = E\frac{k_3}{k_2}e^{ik_3d} \]
\[\Rightarrow A = (-1-(k_3/k_1)E e^{ik_3d})/2\]
Back substituting all values, we have
\[D = A \frac{(k_2 - k_3)k_1}{(k_1+k_3)k_2}\]
\[C = A \frac{(k_2 + k_3)k_1}{(k_1+k_3)k_2}\]
\[B = A \frac{(k_1 - k_3)}{(k_1+k_3)}\]

Transmission Coefficient is given by $\frac{|E|^2k_3}{|A|^2k_1} = 0.75 $
\[\Rightarrow \frac{4k_3}{k_1(1+k_3/k_1)^2} = 0.75\]
\[\Rightarrow \lambda=\frac{3}{16} (1+\lambda)^2\]
\[ \lambda = 3 \textbf{ or } 1/3\]
\[ \frac{9-n}{9}= 9 or \frac {|9-n|}{9} = 1/9\]
\[ n=8 \textbf{ or } -72 \]

These coeffs can be subtituted to get all wavefuntions in terms of amplitude of incident wave.

Now,
\[B = A \frac{(k_1 - k_3)}{(k_1+k_3)}\]
To get the phase change between incident and reflected beam, we consider $Im(B/A)$
\[= Im ((\frac{k_1-k_3}{k_1+k_3})\]
\[= Im( \frac{3-\sqrt(9-n)}{3+\sqrt(9-n)})\]
For n<9 it is 0. For n> 9, we have
\[= Im(\frac{3 - i\sqrt(n-9)}{3 + i\sqrt(n-9)}\]

\[= Im (\frac{(3-i\sqrt(n-9))^2}{n})\]
\[= \frac{-6\sqrt(n-9)}{n}\]

for n=9, above expression gives B=A, hence no phase change


\end{document}

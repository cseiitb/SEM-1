\documentclass{article}
\begin{document}
\title{QM Tutorial Q.82}
\author{Raghav Gupta}
\maketitle
\textbf{If the wave function \(\Psi (r,\theta,\phi,t=0)\) for the case of hydrogen atom is written as a product of three
functions as \(\Psi (r,\theta,\phi,t=0) = R(r)\times \Theta(\theta) \times \Phi (\phi\)) then it can be shown that the radial pert of
the Schrodinger equation can be written as follows\\\\\(\frac{1}{r^2}\frac{d}{dr}(r^2\frac{dR}{dr})+[\frac{2m}{\hbar^2}(E+\frac{e^2}{4\pi \epsilon_0 r}) - \frac{l(l+1)}{r^2}]R = 0\) where the symbols have their usual meanings. For the ground state of hydrogen atom ��=0. For this state\\
(a) Show that \(\Psi(r, t=0) = Ae^{\frac{-r}{a}}\)is a solution of this equation. Find the values of \(A, a\) and the
ground state energy.\\
(b) Calculate the mean distance, root mean square distance and the most probable distance between
the electron and the nucleus in terms of the Bohr radius.\\
(c) What are the classical and quantum mechanical probabilities of finding the electron at \(r>2a\)?}

First we normalize the wave function. Putting \(\[\int_{space} \Psi^*\Psi \mathrm{d}\tau\] = 4\pi A^2\[\int_{0}^{\infty} r^2e^{\frac{-2r}{a}} \mathrm{d}r\] = 1,\) we get \(A = \frac{1}{\sqrt{\pi a^3}}\)\\\\Now, even though we are given the complete wavefunction \(\Psi\) at \(t=0\) instead of just the radial part \(R(r)\), we can still plug in \(\Psi\) in the equation since \(\Psi\) depends only on \(r\), thus its \(\Theta (\theta)\) and \(\Phi (\phi)\) parts are simply constants and will not affect our working since the RHS of the equation is zero.\\\\Putting \(\Psi = \frac{1}{\sqrt{\pi a^3}}e^{\frac{-r}{a}}\) in the equation, we get, for the ground state of Hydrogen atom \((l=0)\), after dividing the LHS by \(\Psi\)\\\\\(\frac{1}{r}(\frac{2me^2}{4\pi \epsilon_0 \hbar^2} - \frac{2}{a})+ (\frac{2mE}{\hbar^2} + \frac{1}{a^2}) = 0\)\\\\As this should, for some value of \(a\), be identically zero, both the coefficient of \(\frac{1}{r}\) and the constant term should be zero.\\\\The former gives us \(a = \frac{4\pi \epsilon_0 \hbar^2}{me^2} = 0.529\AA\), which happens to be the same expression as that of the Bohr radius for the Hydrogen atom.\\Putting this value of \(a\) in the latter condition of the constant term being zero,\\\\\(E = -\frac{me^4}{32\pi^2\epsilon_0^2\hbar^2} = -13.6eV\) which is again the same value as the classically obtained value.\\\\The mean distance of the electron from the centre would be \\\(<r> = \[\int_{space} \Psi^*(r\Psi) \mathrm{d}\tau\] = 4\pi A^2\[\int_{0}^{\infty} r^3e^{\frac{-2r}{a}} \mathrm{d}r\] = \frac{3a}{2}\)\\\\The root mean square distance of the electron from the centre would be\\\(\sqrt{<r^2>} = \sqrt{\[\int_{space} \Psi^*(r^2\Psi) \mathrm{d}\tau\]} = \sqrt{4\pi A^2\[\int_{0}^{\infty} r^4e^{\frac{-2r}{a}} \mathrm{d}r\]} = \sqrt{3}a\)\\\\The most probable distance of the electron is defined as that value of \(r\) for which the probability of the particle being in the range \((r, r+dr)\) is maximized i.e. \(P(r) = |\Psi|^2\frac{dV}{dr}\) is maximized \(\Rightarrow f(r) = r^2e^{\frac{-2r}{a}\) is maximized w.r.t. \(r\). By putting the derivative of this function to 0, we get \(r = 0\) (for minimum) and \(r = a\) (for maximum). Hence the most probable distance of the electron from the centre is \(a\).\\\\Classically, the electron in its ground state, will remain at \(r=a\) and thus have zero probability of being further than \(r=2a\). However, quantum mechanically, this probability will be non zero and will equal 4\pi A^2\[\int_{2a}^{\infty} r^2e^{\frac{-2r}{a}} \mathrm{d}r\] = 0.238.\)
\end{document}
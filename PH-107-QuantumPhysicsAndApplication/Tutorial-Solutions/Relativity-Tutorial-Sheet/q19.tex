\documentclass[10pt, a4paper]{article}
\usepackage[top=80pt, bottom=50pt, left=72pt, right=55pt]{geometry}
\usepackage{enumerate}
\usepackage[fleqn]{amsmath}
\usepackage{graphicx}
\begin{document}
\title{PH-105 Assignment Sheet - 1}
\author{Vaibhav Krishan}
\date{}
\maketitle
\begin{enumerate}
\item[19.]{19. A proton of total energy 2 GeV (2×109 eV) moving along + x direction collides with another proton\\
at rest. If this collision is viewed in the centre of mass frame of reference, it is found that one of the\\
protons moves at an angle of 45 degrees from the initial direction of motion in x-y plane. Find the\\
energy (in GeV) and the x and y components of the momentum of the two protons (in GeV/c) in the\\
centre of mass frame and the original frame. What is the angle which the incoming proton makes after\\
scattering in the original frame with the incident direction of motion? (Take rest mass of proton as 1\\
GeV)\\


{\underline {\bf Solution}} : \\
If energy is 2 GeV and rest mass is 1 GeV then momentum is given by\\
\begin{math} p_{p1} = \sqrt{4-1} / c. \end{math}\\
\begin{math} p_{p1} = \sqrt3 / c.\end{math}\\
First we need to find out the velocity of center of mass of frame in the original frame. Let it be v\begin{math}_{CM}\end{math}.\\
Then using the momentum transformation formula we get in CM frame\\
\begin{math} p'_{p1} = \gamma (\sqrt3/c - v_{CM}*2/c^2) \end{math}\\
and \\
\begin{math} p'_{p2} = -\gamma v_{CM} 1/c^2 \end{math}\\
Now\\
\begin{math} |p'_{p1}| = |p'_{p2}| \end{math} as we are in CM frame of two particles.\\
Using this we get
\begin{math} v_{CM} = c/\sqrt3. \end{math}\\
\begin{math} \gamma = \sqrt{3/2}. \end{math}\\
Now using transformation to CM frame we get\\
\begin{math} p'_{p1} = 1/(\sqrt2 c) = -p'_{p2}\end{math}\\
\begin{math} E'_{p1} = \sqrt{3/2} = E'_{p2}.\end{math}\\
After collision the let x-component of momentum for p1 be p\begin{math} _x\end{math}\\.
Then y component is same and also for p2 x and y components are same but opposite.\\
Using energy conservation we get\\
\begin{math} 2*\sqrt{1+2*p_x^2*c^2} = 2*\sqrt{3/2}.\end{math}\\
\begin{math} p'_x = 1/(2c).\end{math}\\
Using reverse transformation for p1\\
\begin{math} p_x = \sqrt{3/2} (1/(2c) + \sqrt{3/2}*c/(\sqrt3 * c^2))\end{math}\\
\begin{math} p_x = \end{math}\\
and
\begin{math} p_y = p'_y = p'_x = 1/(2c)\end{math}\\
angle to the incident direction is given by
\begin{math} tan \phi = p_y/p_x = 1/(\sqrt(3/2) + \sqrt(3)) \end{math}\\
\begin{math} \phi = 18^o\end{math}\\
}

\end{enumerate}

\end{document}

\documentclass{article}
\usepackage{amsmath}
\begin{document}
\title{Solution to Relativity tutorial Q.17}
\author{Raghav Gupta}
\maketitle

\textbf{Q.17 Find the minimum energy a neutrino must have to initiate the reaction \(\nu + n \rightarrow \tau + p\)}\textbf{ if the target neutron is at rest (rest energy of neutrino = 0 GeV, rest energy of neutron = 1 GeV, rest energy of} \textbf{proton = 1 GeV, rest energy of} \tau = \textbf{2 GeV)}\\\\

For the reactants to possess the minimum energy, we would like the products to possess the minimum energy they can possess (i.e. only their rest mass energy) in the C-frame.\\\\
Let's suppose that the neutrino and the neutron (the reactants) possess energies \(E_\nu\) and \(E_n\) in the C-frame. Also take the neutrino and neutron momenta as \(p_\nu\) and \(p_n\) respectively.\\\\
Conserving energy between the reactants and the products in the C-frame, we get\\
\begin{equation}
\(E_{total_C} = E_\nu + E_n =  3GeV\)
\label{eq:1}
\end{equation}
This is the total energy of the system in the C-frame\\
\\\\Now we use the invariance of \(p_{total}^2 - \frac{E_{total}^2}{c^2}\) for the system across the C-frame and the laboratory frame to transform these energies back to the latter.\\\\Note that the linear momentum in the C-frame is zero and that in the lab frame is only due to the neutrino, hence equaling \(\frac{E_{\nu_l}}{c}\) (where \(E_{\nu_l}\) is the neutrino's energy in the lab frame which we're supposed to find). Also, energy of the neutron in the lab frame is simply its rest mass energy (1 GeV). So, plugging values, we get
\begin{equation}
\(\frac{E_{\nu_l}^2}{c^2} - \frac{(E_{\nu_l} + 1)^2}{c^2}  = -\frac{E_{total_C}^2}{c^2}\)
\label{eq:3}
\end{equation}
Solving, we get \(E_{\nu_l}=4GeV\).\\\\
Thus, \textbf{the minimum energy required by the neutrino to initiate the reaction is 4GeV.}

\end{document}



\documentclass[10pt, a4paper]{article}
\usepackage[top=80pt, bottom=50pt, left=72pt, right=55pt]{geometry}
\usepackage{enumerate}
\usepackage[fleqn]{amsmath}
\usepackage{graphicx}
\begin{document}
\title{PH-105 Assignment Sheet - 1}
\author{Vaibhav Krishan}
\date{}
\maketitle
\begin{enumerate}
\item[8.]{An errant firecracker maniac cruising at \begin{math} 3*10^7 \end{math} m/s in the positive x- direction sets off a firecracker at x= 20 km from the police headquarters (HQ) located at x=0. At the moment the HQ receives the resulting light signal, it instructs a patrol car just then passing in the +x direction under its window and cruising in the same direction at \begin{math} 6*10^7 \end{math} m/s to catch the rogue. Assume that at this moment both HQ and patrol car clocks show time equal to zero. (i) What are the position and time of the firecracker burst in the frame of patrol car? (ii) Find the position of the boy in HQ frame at t=0 and in patrol car frame at t’=0. (iii) Find the time in HQ frame and the patrol car frames when the car reaches the maniac.\\


{\underline {\bf Solution}} : \\
We know that in frame of HQ firecracker burst at x=20 km. Now it was received at t=0 sec. at HQ which means that light travelled for 20 Km. Time taken by
light in HQ frame is given by \\
t = -x / c = \begin{math} -20*10^3 \end{math} / \begin{math} 3*10^8 \end{math} = -200/3 \begin{math} \mu \end{math} s.\\
Now using Lorentz transformation from HQ frame to Patrol car frame we get\\
t` = \begin{math} \gamma (t - xv/c^2) \end{math}\\
where\\
\begin{math} \gamma = 1 / \sqrt{1 - (6*10^7/3*10^8)^2} \end{math}\\
\begin{math} \gamma = 1.02 \end{math}\\
so \\
t` = \begin{math} 1.02 (-200/3 * 10^{-6} - 20*10^3 * 6*10^7/ (9*10^{16})) \end{math}\\
t` = -81.6 \begin{math} \mu \end{math}s\\
x` = 1.02 \begin{math} (20*10^3 - 6*10^7*(-200/3)*10^{-6}) \end{math}\\
x` = 24.49 km.\\
Now in HQ frame for boy x=20 when t = -200/3 \begin{math} \mu \end{math}s.\\
Speed of boy = \begin{math} 3*10^7 \end{math}\\
Distance travelled by boy while light reaches HQ is d = \begin{math} 3*10^7*(-200/3)*10^{-6} \end{math}\\
So\\
\begin{math} x_{boy}|_{HQ} = 20*10^3 + 2*10^3 = 22 Km. \end{math}\\
Using relative velocity between two frames Velocity of boy from car frame is\\
\begin{math} v` = (3*10^7 - 6*10^7) / (1 - 3*10^7*6*10^7 / (3*10^8)^2) = -3.06*10^7 \end{math}\\
time travelled for in car frame is = 81.6 \begin{math} \mu \end{math}s\\
distance travelled in car frame = \begin{math} -3.06*10^7 * 81.6 * 10^{-6} \end{math}\\
final x` = 24.49 - 2.499 = 21.99 Km.\\
In Hq frame distance between car and boy = 22 Km.\\
Relative velocity = \begin{math} 3*10^7 \end{math}\\
time taken to catch = \begin{math} 22*10^3 / 3*10^7 = 733.3 \mu s \end{math}\\
in car frame relative velocity = \begin{math} 3.06*10^7 \end{math}\\
distance = \begin{math} 21.99 Km. \end{math}\\
time taken = \begin{math} 21.99 * 10^3 / 3.06*10^7 \end{math}\\
}
\end{enumerate}

\end{document}

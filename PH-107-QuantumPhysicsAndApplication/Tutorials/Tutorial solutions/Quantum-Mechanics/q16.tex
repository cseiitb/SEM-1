\documentclass[10pt, a4paper]{article}
\usepackage[top=60pt, bottom=50pt, left=72pt, right=55pt]{geometry}
\usepackage{enumerate}
\usepackage[fleqn]{amsmath}
\usepackage{amssymb}
\usepackage{graphicx}
\begin{document}
	\title{PH-105 Assignment Sheet - 2 (Quantum Mechanics)}
	\date{}
	\author{Vaibhav Krishan}
	\maketitle
	\newcommand{\angstrom}{\mbox{\normalfont\AA}}
	\begin{enumerate}
		\item[1.] {\bf Two similar particles of mass m are connected to each other by a spring of
negligible natural length and mass and spring constant k. The particles are made to
rotate in a circle about their common centre of mass, such that the distance
between them is R. Assume that the only force between the particles is the one
provided by the spring. Apply Bohr’s quantization rule to this system and find the
allowed value of r and the energies in terms of fundamental constants if any, the
mass and the spring constant.}\\
		{\underline {\bf Solution}} :\\
		First of all the velocities of particels will be same and they will be rotating around the midpoint of the line joining them.\\
		Writing force equation\\
		\begin{math} kr = mv^2 / (r/2) = 2mv^2/r\end{math} ...(i)\\
		Now angular momentum quatization gives\\
		\begin{math} mvr/2 + mvr/2 = mvr = nh/2\pi\end{math} ...(ii)\\
		Dividing (i) by \begin{math} (ii)^2\end{math} gives\\
		\begin{math} 2/mr^3 = 4\pi^2kr/n^2h^2\end{math}\\
		\begin{math} r^4 = n^2h^2/2mk\pi^2 => r = (n^2h^2/2mk\pi^2)^{1/4}\end{math}\\
		put this in (i) we get\\
		\begin{math} TKE = mv^2 = kr^2 / 2 = k\sqrt{n^2h^2/2mk\pi^2} / 2 = nh/2\pi\sqrt{k/2m}\end{math}\\
		And Potential Energy is\\
		\begin{math} V = 1/2kr^2 = TKE\end{math}\\
		So total energy is\\
		\begin{math} E = nh/\pi\sqrt{k/2m}\end{math}\\
	\end{enumerate}
\end{document} 

\documentclass[10pt, a4paper]{article}
\usepackage[top=60pt, bottom=50pt, left=72pt, right=55pt]{geometry}
\usepackage{enumerate}
\usepackage[fleqn]{amsmath}
\usepackage{amssymb}
\usepackage{graphicx}
\begin{document}
	\title{PH-105 Assignment Sheet - 2 (Quantum Mechanics)}
	\date{}
	\author{Vaibhav Krishan}
	\maketitle
	\newcommand{\angstrom}{\mbox{\normalfont\AA}}
	\begin{enumerate}
		\item[1.] {\bf A photon of energy E is emitted as a result of a particular transition. What would be the value of the recoil energy, assuming that the atom recoils with non relativistic speed. Let the lifetime of the state be of the order of \begin{math} 10^{-8}\end{math} s. What would be the order of natural line width of the emitted line. For what value of E, would the recoil energy be of the same order of magnitude as the natural line width? For order of magnitude calculation take the mass number of the atom as 100. What conclusions would you draw from this regarding resonant absorption?}\\
		{\underline {\bf Solution}} :\\
		By momentum conservation
		\begin{math} p_a = p_p = E/c\end{math} so \begin{math} E_a = p_a^2 / 2m_o = E^2/2m_oc^2\end{math}\\
		Now
		\begin{math} \Delta t = 10^{-8} \end{math} so width of emitted line is \begin{math} \Delta E = \hbar / \Delta t = 6.6 X 10^{-8} eV\end{math}\\
		Now for recoil to be of same order
		\begin{math} E^2 / 2m_oc^2 = 6.6 X 10^{-8}\end{math} gives \begin{math} E = 111 eV\end{math}\\
		This means that as \begin{math} \Delta E / E\end{math} is very small resonant absorption has almost zero probability.\\
	\end{enumerate}
\end{document} 

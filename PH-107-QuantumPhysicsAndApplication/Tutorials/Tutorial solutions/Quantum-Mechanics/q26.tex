\documentclass[10pt, a4paper]{article}
\usepackage[top=60pt, bottom=50pt, left=72pt, right=55pt]{geometry}
\usepackage{enumerate}
\usepackage[fleqn]{amsmath}
\usepackage{amssymb}
\usepackage{graphicx}
\begin{document}
	\title{PH-105 Assignment Sheet - 2 (Quantum Mechanics)}
	\date{}
	\author{Vaibhav Krishan}
	\maketitle
	\newcommand{\angstrom}{\mbox{\normalfont\AA}}
	\begin{enumerate}
		\item[1.] {\bf Show that the Bohr’s condition of quantization of angular momentum leads to a condition of formation of standing wave of electron along the circumference in the Bohr model of hydrogen atom.}\\
		{\underline {\bf Solution}} :\\
		Bohr's quatization rule states that
		\begin{math} mvr = nh/2\pi \end{math} so \begin{math} mv = nh/2\pi r\end{math}\\
		Now for the electron if we find the de-broglie wavelength then
		\begin{math} \lambda = h/p = h/mv = h / (nh/2\pi r) = 2\pi r / n\end{math}\\
		So
		\begin{math} 2\pi r = n\lambda\end{math} which is the condition for constructive interference of a wave if we consider the electron to be a standing wave of wavelength given by de-broglie's hypothesis formed on circumference.\\
	\end{enumerate}
\end{document} 

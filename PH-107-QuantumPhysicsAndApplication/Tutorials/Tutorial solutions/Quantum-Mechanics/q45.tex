\documentclass[10pt, a4paper]{article}
\usepackage[top=60pt, bottom=50pt, left=72pt, right=55pt]{geometry}
\usepackage{enumerate}
\usepackage[fleqn]{amsmath}
\usepackage{amssymb}
\usepackage{graphicx}
\begin{document}
	\title{PH-105 Assignment Sheet - 2 (Quantum Mechanics)}
	\date{}
	\author{Umang Mathur}
	\maketitle
	\begin{enumerate}
		\item[45.] {\bf A charged pi-meson has a rest energy of 140 MeV and a lifetime of 26 ns, while a rho-meson has a rest energy of 765 MeV and a lifetime of $4.4\times 10^{-24}s$. In each case find the absolute and fractional uncertainty in energy. Use the following uncertainty principle for this problem $\Delta E\Delta t \geq \hbar/2$. }\\\\
		{\underline {\bf Solution}} :\\
		By the use of uncertainty principle, we approximate $\Delta E$ as follows:
		\[ \Delta E\approx \frac{\hbar/2}{\Delta t} \]
		where $\Delta t$ is the lifetime of the particle.\\\\
		\begin{enumerate}
			\item {\underline {For the pi-meson}}\\
				\[ \text{Absolute uncertainty in energy} = \Delta E \approx \frac{5.273\times 10^{-35}}{2.6\times 10^{-8}} = 2.028\times 10^{-27} J = 1.268\times 10^{-14} MeV\]\\
				\[ \text{Fractional uncertainty in energy } = \frac{\Delta E}{E} = \frac{1.268\times 10^{-14} MeV}{140 MeV} = 9.057\times 10^{-17}\]
			\item {\underline {For the rho-meson}}\\
				\[ \text{Absolute uncertainty in energy} = \Delta E \approx \frac{5.273\times 10^{-35}}{4.4\times 10^{-24}} = 1.198\times 10^{-11} J = 74.89 MeV\]\\
				\[ \text{Fractional uncertainty in energy } = \frac{\Delta E}{E} = \frac{74.89 MeV}{765 MeV} = 0.098\]
		\end{enumerate}
	\end{enumerate}
\end{document} 

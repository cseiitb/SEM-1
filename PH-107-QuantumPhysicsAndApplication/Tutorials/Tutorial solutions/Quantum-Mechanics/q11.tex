\documentclass[10pt, a4paper]{article}
\usepackage[top=80pt, bottom=50pt, left=72pt, right=55pt]{geometry}
\usepackage{enumerate}
\usepackage[fleqn]{amsmath}
\usepackage{graphicx}
\begin{document}
\title{PH-105 QM Sheet 1}
\date{30.09.2012}
\author{Vipul Singh}
\maketitle
\begin{enumerate}
\item[11.]{\bf Find the energy of the incident x-ray if the maximum kinetic energy of the Compton electron is $m_{0}c^{2}/2.5$?}\\

{\underline {\bf Solution}} : \\
The Compton electron will have maximum kinetic energy when the energy lost by the photon is maximum. That will be the case when the increase in wavelength is maximum. A look at the Compton scattering formula \\
$\lambda^{'}-\lambda=\frac{h}{m_{0}c}(1-cos\theta)$\\
tells us that this occurs at $\theta=\pi$.
So, we have $\theta=\pi$ and $E-E^{'}=0.4m_{0}c^{2}$.\\
$\frac{hc}{E^{'}}-\frac{hc}{E}=\frac{2h}{m_{0}c}$. Replace $E^{'}$ by $E-0.4m_{0}c^{2}$ and solve for E. \\
It reduces to $5E^{2}-2E-1=0$ which implies that $E=0.69m_{0}c^{2}$.
\end{enumerate}
\end{document}

\documentclass{article}
\begin{document}
\title{QM Tutorial Q.9}
\author{Raghav Gupta}
\maketitle
\textbf{Find the smallest energy that a photon may have and still transfer one half of its
energy to an electron initially at rest.}\\\\Suppose the wavelength of the incident photon was \(\lambda_0\). Now, by energy conservation, if it transfers half its energy to the electron, it will be left with half its initial energy or, in other words, its wavelength after being scattered will be double its initial wavelength. Thus if \(\lambda_i = \lambda_0, \lambda_f = 2\times \lambda_0\)\\\\Putting in Compton scattering equation, \(2\lambda_0 - \lambda_0 = \frac{h(1-cos\theta)}{m_0c} \Rightarrow \lambda_0 = \frac{h(1-cos\theta)}{m_0c}\)\\Multiplying \(\frac{c}{\lambda_0}\) on both sides and transposing, we get \(\frac{hc}{\lambda_0} = \frac{m_0c^2}{(1-cos\theta)}\)\\Note that the RHS is the energy of the incident photon.\\ Thus, \(E_{photon} = \frac{m_0c^2}{(1-cos\theta)}\).\\The minimum value \(E_{photon}\) can now have is when \(cos\theta = -1\) i.e. the photon is back-scattered.\\\\Thus, \(E_{photon_{min}} = \frac{m_0c^2}{2}\)\\\\Now, the rest mass energy of the electron being 0.51MeV,\\\\ \(E_{photon_{min}} = \frac{0.51MeV}{2} = 0.255MeV\)
\end{document}
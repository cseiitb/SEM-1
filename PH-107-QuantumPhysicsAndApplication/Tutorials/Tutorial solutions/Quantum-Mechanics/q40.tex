\documentclass{article}
\begin{document}
\title{QM Tutorial Q.40}
\author{Raghav Gupta}
\maketitle
\textbf{A beam of electron of energy \(0.025 eV\) moving along x-direction, passes through a slit of variable width \(w\) placed along y-axis. Estimate the value of the width of the slit for which the spot size on a screen kept at a distance of \(0.5 m\) from slit would be minimum.}\\\\Initially, the particle moved along the x-direction i.e. \(p_y = 0\). But when it passes through the slit, the uncertainty in its y-coordinate suddenly becomes finite, namely, \(\Delta y\) equals the slit width \(w\). Thus, by the uncertainty principle, there must be some uncertainty in its y-momentum over and above its initial y-momentum (which is zero). By the uncertainty principle, we have\\\\\(\Delta y \Delta p_y \geq \frac{\hbar}{2} \Rightarrow \Delta p_y \geq \frac{\hbar}{2w}\)\\\\Thus, the maximum y-momentum a particle passing through the slit can possess (assuming maximum possible precision) is \(|p_y| = \frac{\Delta p_y}{2} = \frac{\hbar}{4w}\) since this momentum arising due to Heisenberg uncertainty can be either in +y or -y direction.\\
\\Now, as the particle traverses \(0.5 m\) (slit-screen distance), it will traverse \(\frac{0.5p_y}{p_x} = \frac{0.5p_y}{\sqrt{2mE}}\) metres in the y-direction (note that the energy of the electron is quite low, hence the classical momentum-energy relation nearly holds).\\\\In addition, a particle may enter the slit at its top or bottom edge, hence the highest y-coordinate (w.r.t. the centre of the screen) that a particle can have upon reaching the screen is\\\\ \(R = \frac{w}{2} + \frac{0.5p_y}{\sqrt{2mE}} = \frac{w}{2} + \frac{0.5\hbar}{4w\sqrt{2mE}}\).\\\\For the minimum of this y-coordinate (same as the radius of the spot formed on the screen), minimize \(R\) w.r.t. \(w\). Putting \(\frac{dR}{dw} = 0\), we get \\\(w = \sqrt{\frac{\hbar}{4\sqrt{2mE}}} \approx 17.56 \mu m\). \\\\Do check that it indeed is a minimum by checking that \(\frac{d^2R}{dw^2} > 0\) for this value of slit width \(w\). 
\end{document}
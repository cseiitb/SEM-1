\documentclass[10pt, a4paper]{article}
\usepackage[top=60pt, bottom=50pt, left=72pt, right=55pt]{geometry}
\usepackage{enumerate}
\usepackage[fleqn]{amsmath}
\usepackage{amssymb}
\usepackage{graphicx}
\begin{document}
	\title{PH-105 Assignment Sheet - 2 (Quantum Mechanics)}
	\date{}
	\author{Vaibhav Krishan}
	\maketitle
	\newcommand{\angstrom}{\mbox{\normalfont\AA}}
	\begin{enumerate}
		\item[1.] {\bf Calculate the wavelength of the matter waves associated with the following. Compare in each case the result with the respective dimension of the object. In which case will it be possible to observe the wave nature.
(a) A 2000 kg car moving with a speed of 100km/h.
(b) A 0.28 kg cricket ball moving with a speed of 40 m/s.
(c) An electron moving a speed of \begin{math} 10^7 \end{math} m/s.}\\
		{\underline {\bf Solution}} :\\
		(a) 
		First changing velocity to SI units we get
		\begin{math} v = 500/18\end{math} m/s\\
		Now
		\begin{math} \lambda = 6.6 X 10^{-34} / 2 X 10^3 X 27.78 = 1.1 X 10^{-38}\end{math} m.\\
		(b)
		\begin{math} \lambda = 6.6 X 10^{-34} / 0.28  X 40 = 5.8 X 10^{-35}\end{math} m.\\
		(c)
		\begin{math} \lambda = 6.6 X 10^{-34} / 9.1X10^{-31} X 10^7 = 0.73 \AA \end{math}\\
		As the wavelength of massless waves is of minimum order of \begin{math} 10^{-15} \end{math} so ball and car will have very small wave component i.e. unobservable while for electron it can be observed.\\
	\end{enumerate}
\end{document} 

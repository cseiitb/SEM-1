\documentclass{article}
\begin{document}
\title{QM Tutorial Q.14}
\author{Raghav Gupta}
\maketitle
\textbf{Consider Compton Scattering. Show that if the angle of scattering \(\theta\) increases
beyond a certain value \(\theta_0\), the scattered photon will never have energy larger than
\(2m_oc^2\), irrespective of the energy of the incident photon. Find the value of \(\theta_0\).}\\\\
Consider the equation relating the wavelengths of incident and scattered photon-\\\ \(\lambda_f - \lambda_i = \frac{h(1-cos\theta)}{m_0c}\)\\Using the relation \(E_{photon} = \frac{hc}{\lambda}\), we replace the photon wavelengths by their energies and divide by \(hc\) to get\\\\
\(\frac{1}{E_f} - \frac{1}{E_i} = \frac{(1-cos\theta)}{m_0c^2}\)
Rearranging, we get\\\\
\(E_f = \frac{E_i\times m_0c^2}{E_i\times (1-cos\theta) + m_0c^2}\)\\\\Dividing the numerator and denominator by \(E_i\), we get\\\\\(E_f = \frac{m_0c^2}{(1-cos\theta) + \frac{m_0c^2}{E_i}}\)\\\\Clearly, the maximum value for a fixed \(\theta\) this expression can have is \(\frac{m_0c^2}{1-cos\theta}\), that is when \(E_i \rightarrow \infty\) (since \(E_i > 0\)). Thus, no matter what the initial energy of the photon may be, there is a value of \(\theta = \theta_0\) for which \(E_f < 2m_0c^2\). To calculate \(\theta_0\), we put\\\\\(E_f = \frac{m_0c^2}{1-cos\theta} < 2m_0c^2 \Rightarrow \theta > 60^{\circ}\)\\\\Thus, \underline{\(\theta_0 = 60^{\circ}\)}

\end{document}

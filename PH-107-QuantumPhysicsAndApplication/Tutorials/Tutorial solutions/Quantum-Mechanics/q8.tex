\documentclass{article}
\begin{document}
\title {QM Tutorial Q.8}
\author{Raghav Gupta}
\maketitle
\textbf{A photon strikes an electron at rest and undergoes a pair production process, giving rise to an electron and a positron, i.e.\\\\\(\gamma + e^- \rightarrow e^- + e^+ + e^-\)\\\\The two electrons and the positron move off with identical momenta in the same direction as the incident photon. Find the energy of the photon and the speed of the particles.}\\\\Suppose the initial energy of the photon was \(E\) and the final momentum of each particle after the pair production be \(p_f\). Conserving momentum before and after the process\\\\\(\frac{E}{c} = 3p_f\)\\\\Conserving energy before and after the interaction (using, for the particles formed post-interaction, the relation \(E = \sqrt{p^2c^2 + m_0^2c^4}\))\\\\\(E + m_0c^2 = 3\sqrt{p_f^2c^2 + m_0^2c^4}\)\\\\Replacing \(p_f = \frac{E}{3c}\) from the above equation and squaring both sides, we get\\\\\(E = 4m_0c^2 = 4\times 0.51MeV = 4.04MeV\)\\\\Now, the energy of each particle created after the process, say \(E_p= \sqrt{p_f^2c^2 + m_0^2c^4}\)\\\\Replacing \(p_f = \frac{E}{3c}\) using the calculated value of \(E\), writing \(E_p = \gamma m_0c^2\) and comparing, we get\\\\\(\gamma = \frac{5}{3} \Rightarrow\) Speed of particles post-interaction \(= 0.8c\).

\end{document}
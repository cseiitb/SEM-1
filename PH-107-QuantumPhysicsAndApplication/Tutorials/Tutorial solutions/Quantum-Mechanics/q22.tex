\documentclass[10pt, a4paper]{article}
\usepackage[top=60pt, bottom=50pt, left=72pt, right=55pt]{geometry}
\usepackage{enumerate}
\usepackage[fleqn]{amsmath}
\usepackage{amssymb}
\usepackage{graphicx}
\begin{document}
	\title{PH-105 Assignment Sheet - 2 (Quantum Mechanics)}
	\date{}
	\author{Vaibhav Krishan}
	\maketitle
	\newcommand{\angstrom}{\mbox{\normalfont\AA}}
	\begin{enumerate}
		\item[1.] {\bf (a) A source of photons of frequency \begin{math} \nu \end{math} is moving with a speed v in laboratory frame of reference. Show that in the limit v is very small then c, the frequency of photon \begin{math} \nu' \end{math} , as observed in laboratory frame of reference is given by the following expression. \begin{math} \nu' = \nu(1+v/c) \end{math}.\\
(b) What is the value of the required speed in case the energy of photons of energy 14.4 keV is to be increased by \begin{math} 10^{-6} \end{math} eV?}\\
		{\underline {\bf Solution}} :\\
		(a) We know by doppler effect that\\
		\begin{math} \nu' = \nu\sqrt{1+\beta/1-\beta} \beta = v/c \end{math}\\
		Now using approximations we get\\
		\begin{math} (1+v/c)^{1/2} ~ 1+v/2c\end{math} and\\
		\begin{math} (1-v/c)^{-1/2} ~ 1+v/2c\end{math}\\
		Now\\
		\begin{math} (1+v/2c)^2 = 1+v/c + (v/2c)^2 \end{math} but \begin{math} (v/2c)^2 \end{math} is very small compared to rest of expression so\\
		\begin{math} \nu' = \nu\sqrt{1+\beta/1-\beta} ~ \nu(1+v/c)\end{math}\\
		(b) \begin{math} \Delta E = 10^{-6} eV \end{math} \begin{math} E_i = 14.4 KeV \end{math}\\
		\begin{math} \nu'/\nu = E'/E = 1+\Delta E/E = 1+v/c\end{math}\\
		\begin{math} v/c = \Delta E/E = 10^{-6}/14.4*10^3 = 10^{-9}/14.4\end{math}\\
		\begin{math} v = 0.3/14.4 \end{math} m/s\\
	\end{enumerate}
\end{document} 

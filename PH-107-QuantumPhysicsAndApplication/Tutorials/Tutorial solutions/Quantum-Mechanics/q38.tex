\documentclass[10pt, a4paper]{article}
\usepackage[top=60pt, bottom=50pt, left=72pt, right=55pt]{geometry}
\usepackage{enumerate}
\usepackage[fleqn]{amsmath}
\usepackage{amssymb}
\usepackage{graphicx}
\begin{document}
	\title{PH-105 Assignment Sheet - 2 (Quantum Mechanics)}
	\date{}
	\author{Umang Mathur}
	\maketitle
	\begin{enumerate}
		\item[38.] {\bf A wave packet is constructed by superposing waves, their wavelengths varying continuously in the following way:\\
				\[ y(x,t) = \int A(k)\cos{(kx-\omega t)} \,dk \] \\
			Where A(k) = A for $(k_{o} - \Delta k/2) \leq k_{o} \leq k_{o} + \Delta k/2)$ and = 0 otherwise.\\
			Sketch approximately y(x,t) and estimate $\Delta x$ by taking the difference between two values of $x$ for which the central maximum and nearest minimum is observed in the envelope. Verify uncertainty principle from this. }\\\\
		{\underline {\bf Solution}} :\\
		
		Integrating between $k = k_{o} - \Delta k/2$ and $k = k_{o} + \Delta k/2$, we have,\\
		 		\[ y(x,t) = \left. \frac{A}{x} \sin{(kx-\omega t)} \right|_{k_{o} - \Delta k/2}^{k_{o} + \Delta k/2} = \frac{2A}{x} \sin{(\frac{\Delta k}{2}x)}\cos{(k_{o}x - \omega t)} \]\\
		 		
		The function can be plotted as follows:
		
		\includegraphics{graph.png}
		
		The envelope curve is given by:-
				\[ \xi(x) = \frac{2A}{x}\sin{(\frac{\Delta k}{2}x)} \] \\
		Central maxima occurs at $x = 0$. For neares minimum, we differentiate $\xi(x)$ to find the extremum:
				\[ \frac{\partial \xi(x)}{\partial x} = 0\]
				\[ 2A(\frac{x\frac{\Delta k}{2}\cos{(\frac{\Delta k}{2}x)} - \sin{(\frac{\Delta k}{2}x)}}{x^{2}}) = 0 \]
				\[ x\frac{\Delta k}{2} = \tan{(x\frac{\Delta k}{2})} \]
		Let $x_{o}$ be the solution of the above equation. Then $\Delta x = x_{o} - 0 = x_{o}$. The solution of the equation can be found using analytical methods. The value of $x_{o}$ thus is $\frac{8.98682}{\Delta k}$.\\
		Now, $\Delta p = \hbar \Delta k$.\\
		Thus, the product $ \Delta x \Delta p = 8.986 \hbar > \hbar / 2$.\\
		This verifies the uncertainty principle.
				
	\end{enumerate}
\end{document} 
